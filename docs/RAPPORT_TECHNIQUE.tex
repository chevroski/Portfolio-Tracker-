\documentclass[11pt,a4paper]{article}

% ========== PACKAGES ==========
\usepackage[utf8]{inputenc}
\usepackage[T1]{fontenc}
\usepackage[french]{babel}
\usepackage{geometry}
\usepackage{graphicx}
\usepackage{xcolor}
\usepackage{listings}
\usepackage{booktabs}
\usepackage{hyperref}
\usepackage{fancyhdr}
\usepackage{titlesec}
\usepackage{enumitem}
\usepackage{tcolorbox}
\usepackage{fontawesome5}
\usepackage{tabularx}
\usepackage{float}
\usepackage{caption}
\usepackage{subcaption}

% ========== MISE EN PAGE ==========
\geometry{margin=2cm, top=2.5cm, bottom=2.5cm}

% ========== COULEURS ==========
\definecolor{primary}{RGB}{41, 128, 185}
\definecolor{secondary}{RGB}{44, 62, 80}
\definecolor{success}{RGB}{39, 174, 96}
\definecolor{codebackground}{RGB}{248, 249, 250}
\definecolor{codeborder}{RGB}{222, 226, 230}

% ========== STYLE HYPERREF ==========
\hypersetup{
    colorlinks=true,
    linkcolor=primary,
    urlcolor=primary,
    pdftitle={Rapport Technique - PortfolioTracker},
    pdfauthor={Adam Houri}
}

% ========== STYLE CODE ==========
\lstdefinestyle{javastyle}{
    backgroundcolor=\color{codebackground},
    basicstyle=\ttfamily\small,
    keywordstyle=\color{primary}\bfseries,
    stringstyle=\color{success},
    commentstyle=\color{codeborder}\itshape,
    breaklines=true,
    frame=single,
    rulecolor=\color{codeborder},
    numbers=left,
    numberstyle=\tiny\color{gray},
    numbersep=8pt,
    tabsize=4,
    showstringspaces=false
}
\lstset{style=javastyle}

% ========== STYLE TITRES ==========
\titleformat{\section}{\Large\bfseries\color{primary}}{\thesection}{1em}{}[\titlerule]
\titleformat{\subsection}{\large\bfseries\color{secondary}}{\thesubsection}{1em}{}

% ========== EN-TÊTE / PIED DE PAGE ==========
\pagestyle{fancy}
\fancyhf{}
\rhead{\textcolor{gray}{II.1102 — Programmation Java}}
\lhead{\textcolor{gray}{PortfolioTracker}}
\cfoot{\textcolor{gray}{\thepage}}
\renewcommand{\headrulewidth}{0.4pt}
\renewcommand{\footrulewidth}{0.4pt}

% ========== BOÎTES COLORÉES ==========
\tcbuselibrary{skins,breakable}
\newtcolorbox{infobox}[1][]{
    colback=primary!5,
    colframe=primary,
    fonttitle=\bfseries,
    title=#1,
    breakable
}

\newtcolorbox{successbox}[1][]{
    colback=success!5,
    colframe=success,
    fonttitle=\bfseries,
    title=#1,
    breakable
}

% ========== DOCUMENT ==========
\begin{document}

% ========== PAGE DE TITRE ==========
\begin{titlepage}
    \centering
    \vspace*{2cm}
    
    {\Huge\bfseries\textcolor{primary}{PortfolioTracker}\par}
    \vspace{0.5cm}
    {\Large Application de suivi de portefeuilles financiers\par}
    
    \vspace{2cm}
    
    {\LARGE\bfseries Rapport Technique\par}
    
    \vspace{2cm}
    
    \begin{tcolorbox}[colback=primary!5, colframe=primary, width=0.7\textwidth]
        \centering
        \textbf{Module :} II.1102 — Programmation Java Avancée\\[0.3cm]
        \textbf{Auteur :} Adam Houri\\[0.3cm]
        \textbf{Date :} Janvier 2026
    \end{tcolorbox}
    
    \vfill
    
    \includegraphics[width=0.8\textwidth]{exemple_portfolio.png}
    
    \vfill
\end{titlepage}

% ========== TABLE DES MATIÈRES ==========
\tableofcontents
\newpage

% ========== SECTION 1 : INTRODUCTION ==========
\section{Introduction}

PortfolioTracker est une application desktop \textbf{JavaFX} permettant le suivi de portefeuilles financiers (cryptomonnaies et actions). Elle offre une visualisation en temps réel des positions avec graphiques, import CSV, et fonctionnalités d'analyse avancées.

\begin{table}[H]
\centering
\renewcommand{\arraystretch}{1.3}
\begin{tabularx}{\textwidth}{lX}
\toprule
\textbf{Aspect} & \textbf{Choix technique} \\
\midrule
\faCode\ Architecture & MVC + Services (Pattern Singleton) \\
\faDesktop\ Interface & JavaFX 21 (FXML + CSS) \\
\faDatabase\ Persistance & JSON local (Gson) \\
\faGlobe\ APIs & Binance, Yahoo Finance, ExchangeRate \\
\faLock\ Sécurité & Chiffrement XOR (pédagogique) \\
\bottomrule
\end{tabularx}
\caption{Synthèse des choix techniques}
\end{table}

% ========== SECTION 2 : FONCTIONNALITÉS ==========
\section{Fonctionnalités implémentées}

\subsection{Fonctionnalités principales}

\begin{successbox}[\faCheckCircle\ Fonctionnalités livrées]
\begin{itemize}[noitemsep]
    \item \textbf{Gestion multi-portfolios} : création, suppression, clonage, changement de devise
    \item \textbf{Gestion d'assets} : ajout/suppression, transactions BUY/SELL/REWARD/CONVERT
    \item \textbf{Graphiques} : évolution valeur (1W/1M/3M/1Y), allocation pie chart, Compare All
    \item \textbf{Import CSV Coinbase} : parsing automatique des transactions
    \item \textbf{Multi-devises} : EUR, USD, GBP, CHF, JPY
\end{itemize}
\end{successbox}

\subsection{Fonctionnalités avancées}

\begin{itemize}[noitemsep]
    \item \textbf{Events sur graphiques} : marqueurs crash/hack/décision personnelle
    \item \textbf{Analysis} : Profit vs Loss Days, Best/Worst Day (30 jours)
    \item \textbf{Whale Alerts} : transactions crypto > \$1M des dernières 24h
    \item \textbf{Chiffrement local} : activation via passphrase au démarrage
\end{itemize}

\begin{figure}[H]
    \centering
    \includegraphics[width=0.9\textwidth]{partie_charts.png}
    \caption{Vue Charts : évolution de la valeur et allocation}
\end{figure}

% ========== SECTION 3 : ARCHITECTURE ==========
\section{Architecture logicielle}

Le projet suit une architecture \textbf{MVC (Model-View-Controller)} avec une couche Service indépendante.

\begin{figure}[H]
    \centering
    \includegraphics[width=0.95\textwidth]{diagramme_mvc.png}
    \caption{Diagramme d'architecture MVC}
\end{figure}

\begin{table}[H]
\centering
\renewcommand{\arraystretch}{1.3}
\begin{tabularx}{\textwidth}{lX}
\toprule
\textbf{Package} & \textbf{Responsabilité} \\
\midrule
\texttt{com.portfoliotracker.model} & Entités métier (Portfolio, Asset, Transaction) \\
\texttt{com.portfoliotracker.controller} & Orchestration UI, gestion événements \\
\texttt{com.portfoliotracker.service} & Logique métier (Singleton) \\
\texttt{com.portfoliotracker.api} & Clients HTTP externes \\
\bottomrule
\end{tabularx}
\caption{Organisation des packages}
\end{table}

\textbf{Services implémentés :} PortfolioService, MarketDataService, PersistenceService, EncryptionService, CacheService, EventService, AnalysisService, DemoService.

% ========== SECTION 4 : MODÈLE DE DONNÉES ==========
\section{Modèle de données}

\subsection{Portfolio}
Conteneur principal avec : \texttt{id}, \texttt{name}, \texttt{description}, \texttt{currency}, \texttt{createdAt}, et une liste d'\texttt{Asset}.

\subsection{Asset}
Actif financier avec : \texttt{ticker}, \texttt{name}, \texttt{type} (CRYPTO/STOCK), et liste de \texttt{Transaction}.

\textbf{Méthodes de calcul :} \texttt{getTotalQuantity()}, \texttt{getAverageBuyPrice()}, \texttt{getTotalInvested()}

\subsection{Transaction}
Opération avec : \texttt{type} (BUY/SELL/REWARD/CONVERT), \texttt{quantity}, \texttt{pricePerUnit}, \texttt{date}, \texttt{fees}.

% ========== SECTION 5 : SERVICES ==========
\section{Services clés}

\subsection{Pattern Singleton}

\begin{lstlisting}[language=Java, caption=Implémentation Singleton]
public static PortfolioService getInstance() {
    if (instance == null) {
        instance = new PortfolioService();
    }
    return instance;
}
\end{lstlisting}

\subsection{Cache à deux niveaux}

Le \texttt{MarketDataService} implémente un cache optimisé :
\begin{itemize}[noitemsep]
    \item \textbf{Niveau 1 — Mémoire} : \texttt{ConcurrentHashMap} avec TTL de 60 secondes
    \item \textbf{Niveau 2 — Disque} : fichiers JSON persistants pour les prix USD
\end{itemize}

% ========== SECTION 6 : INTERFACE ==========
\section{Interface utilisateur}

\begin{table}[H]
\centering
\renewcommand{\arraystretch}{1.3}
\begin{tabularx}{\textwidth}{llX}
\toprule
\textbf{Écran} & \textbf{Fichier FXML} & \textbf{Description} \\
\midrule
Main & main.fxml & Toolbar + navigation \\
Portfolio & portfolio-view.fxml & Liste des assets + P\&L \\
Charts & chart-view.fxml & Courbes + allocation \\
Analysis & analysis-view.fxml & Whale Alerts + statistiques \\
\bottomrule
\end{tabularx}
\caption{Écrans de l'application}
\end{table}

\begin{figure}[H]
    \centering
    \includegraphics[width=0.85\textwidth]{partie_analyse.png}
    \caption{Vue Analysis : Whale Alerts et Portfolio Analysis}
\end{figure}

% ========== SECTION 7 : CONCURRENCE ==========
\section{Concurrence \& Performance}

Les appels réseau sont exécutés dans des \textbf{Tasks JavaFX} pour ne pas bloquer l'interface :

\begin{lstlisting}[language=Java, caption=Chargement asynchrone]
Task<Map<String, Double>> task = new Task<>() {
    @Override
    protected Map<String, Double> call() {
        // Appels API en arriere-plan
        return prices;
    }
};
task.setOnSucceeded(e -> updateUI());
new Thread(task).start();
\end{lstlisting}

% ========== SECTION 8 : PERSISTANCE ==========
\section{Persistance \& Chiffrement}

\subsection{Structure de stockage}

\begin{verbatim}
data/
├── portfolios/     # JSON (ou .enc si chiffré)
├── cache/          # Prix historiques
└── events/         # Événements utilisateur
\end{verbatim}

\subsection{Chiffrement XOR}

\begin{infobox}[\faExclamationTriangle\ Note pédagogique]
Le chiffrement XOR est une implémentation simplifiée à but éducatif. Une application de production nécessiterait AES-256.
\end{infobox}

\begin{figure}[H]
    \centering
    \includegraphics[width=0.5\textwidth]{passphrase.png}
    \caption{Dialog de passphrase au démarrage}
\end{figure}

% ========== SECTION 9 : TESTS ==========
\section{Tests unitaires}

\begin{table}[H]
\centering
\renewcommand{\arraystretch}{1.3}
\begin{tabular}{lcl}
\toprule
\textbf{Classe de test} & \textbf{Nb tests} & \textbf{Objectif} \\
\midrule
AssetTest & 5 & Calculs financiers (quantité, prix moyen) \\
EncryptionServiceTest & 6 & Round-trip encrypt/decrypt \\
\bottomrule
\end{tabular}
\caption{Couverture des tests}
\end{table}

\begin{verbatim}
$ mvn test
Tests run: 11, Failures: 0, Errors: 0
BUILD SUCCESS
\end{verbatim}

% ========== SECTION 10 : CONCLUSION ==========
\section{Conclusion}

\begin{successbox}[\faCheckCircle\ Objectifs atteints]
\begin{itemize}[noitemsep]
    \item Gestion multi-portfolios avec visualisation graphique
    \item Import CSV Coinbase et APIs temps réel
    \item Fonctionnalités avancées : Analysis, Whale Alerts, Encryption
    \item Architecture MVC propre, testée et documentée
\end{itemize}
\end{successbox}

\subsection{Limites et évolutions}

\begin{table}[H]
\centering
\renewcommand{\arraystretch}{1.3}
\begin{tabularx}{\textwidth}{lX}
\toprule
\textbf{Limite actuelle} & \textbf{Évolution envisagée} \\
\midrule
Chiffrement XOR simplifié & Implémenter AES-256 \\
Cache limité à USD & Étendre aux autres devises \\
Pas de tests UI & Ajouter TestFX \\
\bottomrule
\end{tabularx}
\caption{Pistes d'amélioration}
\end{table}

\end{document}
